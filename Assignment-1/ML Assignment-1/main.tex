\documentclass[11pt]{article}
\usepackage[margin=1in]{geometry}
\usepackage{enumitem}
\usepackage{hyperref}
\usepackage{graphicx}
\usepackage{array}
\usepackage{multicol}
\usepackage{longtable}
\usepackage{titlesec}
\usepackage{amsmath} 
\usepackage{float} % Required for [H] placement

\begin{document}

\begin{center}
    \large \textbf{Sri Sivasubramaniya Nadar College of Engineering, Chennai} \\
    (An autonomous Institution affiliated to Anna University) \\
    \vspace{0.3cm}
\end{center}

\begin{table}[H] 
\renewcommand{\arraystretch}{1.5}
\resizebox{\textwidth}{!}{%
\begin{tabular}{|l|cll|}
\hline
Degree \& Branch     & \multicolumn{1}{c|}{B.E.   Computer Science \& Engineering} & \multicolumn{1}{l|}{Semester}        & V                          \\ \hline
Subject Code \& Name & \multicolumn{3}{c|}{UCS2612 \& Machine Learning Algorithms Laboratory}                                                         \\ \hline
Academic year       & \multicolumn{1}{c|}{2025-2026 (Even)}                         & \multicolumn{1}{c|}{Batch:2023-2027} & \multicolumn{1}{c|}{\textbf{Due date: 23/12/25}} \\ \hline
\end{tabular}%
}
\end{table}

\begin{center}
 \textbf{Experiment 1: Working with Python packages -- Numpy, Scipy, Scikit-Learn, Matplotlib}
\end{center}
\vspace{0.5cm}
{
\raggedleft
Name: Melvin Isaac I\\
Reg.No: 3122235001082\\
}
\vspace{0.5cm}
\noindent \textbf{Aim:} \\
To explore and work with Python packages like Numpy, Scikit-learn, and Matplotlib on datasets from public repositories and identify ML tasks, feature selection techniques, and suitable algorithms.

\vspace{0.5cm}
\noindent
\textbf{Libraries used:}
\begin{itemize}[noitemsep]
    \item Numpy (imported as \texttt{np})
    \item Pandas (imported as \texttt{pd})
    \item Matplotlib.pyplot (imported as \texttt{plt})
    \item Seaborn (imported as \texttt{sns})
    \item OpenCV (\texttt{cv2})
    \item Standard Scaler(from sklearn.preprocessing)
    \item Math (Standard Library)
    
\end{itemize}

\vspace{0.5cm}
\noindent
\textbf{Mathematical and Theoretical description of the objectives performed:}

\begin{itemize}
    \item \textbf{Purpose:} Perform Exploratory Data Analysis (EDA) to understand data behavior, summarize value distributions, identify anomalies, evaluate class proportions, and uncover inter-feature relationships.
     
    \item \textbf{Summary Statistics}
    \begin{itemize}
        \item Mean value computed as: $\mu = \frac{1}{n} \sum x_i$
        \item Sample variance expressing dispersion: $s^2 = \frac{1}{n-1} \sum (x_i - \mu)^2$
        \item Quantile measures (Median, $Q1$, $Q3$) used to analyze spread and construct boxplots.
    \end{itemize}
     
    \item \textbf{Distribution Analysis Using Histograms}
    \begin{itemize}
        \item Frequency-based approximation of empirical distributions through binning to reveal shape and central tendencies.
        \item Kernel Density Estimation (KDE) overlaid to provide a smooth estimate of the underlying distribution:
        \[
        \hat{f}(x) = \frac{1}{nh} \sum_{i=1}^{n} K\left(\frac{x - x_i}{h}\right)
        \]
        where $K$ denotes the kernel function and $h$ represents the bandwidth parameter.
    \end{itemize}
     
    \item \textbf{Boxplot Visualization}
    \begin{itemize}
        \item Displays median and quartiles ($Q1$, $Q3$) with the interquartile range defined as $IQR = Q3 - Q1$.
        \item Whiskers extend to data points within $[Q1 - 1.5 \cdot IQR, \ Q3 + 1.5 \cdot IQR]$, while values beyond this range are flagged as outliers.
    \end{itemize}
     
    \item \textbf{Feature Relationship Plots}
    \begin{itemize}
        \item Pairwise scatter visualizations are used to detect trends, separability between classes, and nonlinear dependencies.
        \item Diagonal plots present KDEs or histograms to highlight individual feature distributions.
    \end{itemize}
     
    \item \textbf{Correlation Visualization}
    \begin{itemize}
        \item Linear association between variables quantified using the Pearson correlation coefficient:
        \[
        r_{XY} = \frac{\text{cov}(X,Y)}{\sigma_X \sigma_Y}
        \]
        \[
        \text{cov}(X,Y) = \frac{1}{n-1} \sum (X_i - \bar{X})(Y_i - \bar{Y})
        \]
        \item Coefficient values lie in the interval $[-1, 1]$, indicating both magnitude and direction of linear dependence.
    \end{itemize}
     
    \item \textbf{Categorical Data Visualization}
    \begin{itemize}
        \item Count-based bar plots are used to examine category frequencies, with optional class-wise grouping to study imbalance patterns.
    \end{itemize}

    \item \textbf{Image Data Exploration}
    \begin{itemize}
        \item Histograms or KDE plots of image dimensions (height and width) are used to determine consistency and preprocessing requirements.
        \item Representative image samples are displayed to qualitatively assess visual characteristics.
    \end{itemize}
     
    \item \textbf{Handling Missing Data}
    \begin{itemize}
        \item Column-wise inspection of missing entries to inform strategies such as imputation or exclusion.
    \end{itemize}

    \item \textbf{Analytical Objectives}
    \begin{itemize}
        \item Detect skewed distributions, multiple modes, anomalous observations, correlated features, and class imbalance in order to support informed preprocessing and model selection.
    \end{itemize}
\end{itemize}


\vspace{0.5 cm}
\noindent
\textbf{Results and Discussions:}
\subsection{Loan Amount Prediction}

\begin{figure}[H]
    \centering
    \includegraphics[width=0.75\linewidth]{loan_columns.png}
    \caption{Dataset Columns}
\end{figure}

\begin{figure}[H]
    \centering
    \includegraphics[width=\linewidth]{loan_distribution.png}
    \caption{Loan Amount Distribution(histogram plot)}
\end{figure}

\begin{figure}[H]
    \centering
    \includegraphics[width=\linewidth]{loan_boxplot.png}
    \caption{Boxplot Distribution}
\end{figure}

\begin{figure}[H]
    \centering
    \includegraphics[width=0.75\linewidth]{loan_correlation.png}
    \caption{Correlation Matrix}
\end{figure}
\subsection{Predicting Diabetes}

\begin{figure}[H]
    \centering
    \includegraphics[width=0.75\linewidth]{diabetes_columns.png}
    \caption{Dataset Columns}
\end{figure}

\begin{figure}[H]
    \centering
    \includegraphics[width=\linewidth]{diabetes_distribution.png}
    \caption{Histogram Distribution}
\end{figure}

\begin{figure}[H]
    \centering
    \includegraphics[width=\linewidth]{diabetes_boxplot.png}
    \caption{Boxplot distribution}
\end{figure}

\begin{figure}[H]
    \centering
    \includegraphics[width=0.75\linewidth]{diabetes_correlation.png}
    \caption{Correlation Matrix}
\end{figure}
\subsection{Classification of Email Spam}
\begin{figure}[H]
    \centering
    \includegraphics[width=0.75\linewidth]{spam_ham_columns.png}
    \caption{dataset columns}
\end{figure}

\begin{figure}[H]
    \centering
    \includegraphics[width=0.75\linewidth]{spam_ham_features.png}
    \caption{spam vs ham distribution}
\end{figure}

\begin{figure}[H]
    \centering
    \includegraphics[width=\linewidth]{word_count_distirubtion.png}
    \caption{word count distribution(Normalisation)}
\end{figure}

\begin{figure}[H]
    \centering
    \includegraphics[width=\linewidth]{word_count_distirubtion_1.png}
    \caption{word count distribution(Standardisation)}
\end{figure}

\begin{figure}[H]
    \centering
    \includegraphics[width=0.75\linewidth]{spam_correlation.png}
    \caption{Correlation Matrix}
\end{figure}
\begin{figure}[H]
    \centering
    \includegraphics[width=0.75\linewidth]{word_scatterplot.png}
    \caption{Correlation Matrix}
\end{figure}

\subsection{Handwritten Character Recognition (MNIST)}

\begin{figure}[H]
    \centering
    \includegraphics[width=0.75\linewidth]{sample_images.png}

    \caption{sample images}
\end{figure}


\begin{figure}[H]
    \centering
    \includegraphics[width=\linewidth]{minst_distribution.png}
    \caption{Digit Distribution}
\end{figure}

\begin{figure}[H]
    \centering
    \includegraphics[width=\linewidth]{mnist_pixel_intensity.png}
    \caption{Pixel Intensity Analysis}
\end{figure}

\begin{figure}[H]
    \centering
    \includegraphics[width=0.75\linewidth]{mnist_hw.png}

    \caption{Height weight distribution}
\end{figure}


\subsection{Iris Dataset}

\begin{figure}[H]
    \centering
    \includegraphics[width=0.75\linewidth]{iris_columns.png}
    \caption{Dataset Columns}
\end{figure}

\begin{figure}[H]
    \centering
    \includegraphics[width=\linewidth]{iris_hist_box.png}
    \caption{Histogram and Boxplot Distribution}
\end{figure}

\begin{figure}[H]
    \centering
    \includegraphics[width=\linewidth]{iris_class.png}
    \caption{class distribution}
\end{figure}

\begin{figure}[H]
    \centering
    \includegraphics[width=0.75\linewidth]{iris_correlation.png}
    \caption{Correlation Matrix}
\end{figure}



\begin{table}[H] 
\centering
\vspace{0.5cm}
\renewcommand{\arraystretch}{1.5}
\begin{tabular}{|p{0.15\textwidth}|p{0.22\textwidth}|p{0.28\textwidth}|p{0.25\textwidth}|}
\hline
\textbf{Dataset} & \textbf{Type of ML Task} & \textbf{Feature Selection Technique} & \textbf{Suitable ML Algorithm} \\ \hline
Iris Dataset & Multi-class Classification & Correlation Matrix / ANOVA F-value & k-Nearest Neighbors (k-NN), Decision Trees \\ \hline
Loan Amount Prediction & Regression & Recursive Feature Elimination (RFE) / Pearson Correlation & Linear Regression, Random Forest Regressor \\ \hline
Predicting Diabetes & Binary Classification & Chi-Square Test / SelectKBest & Logistic Regression, Support Vector Machine (SVM) \\ \hline
Classification of Email Spam & Binary Classification (NLP) & Information Gain / Chi-Square (on word vectors) & Naive Bayes, SVM \\ \hline
Handwritten Character Recognition / MNIST & Multi-class Image Classification & Principal Component Analysis (PCA) & Convolutional Neural Networks (CNN), SVM \\ \hline
\end{tabular}
\end{table}

\vspace{0.5cm}
\noindent
\textbf{Learning Practices:}

\begin{itemize}
    \item Examine dataset organization: Understand the structure by analyzing dimensions, data types, and missing entries.
\item Explore data distributions: Develop the ability to visualize patterns using histograms, box plots, and correlation maps.
\item Assess class proportions: Analyze label balance and its influence on model behavior and outcomes.
\item Analyze feature interactions: Investigate relationships among variables through pair plots and correlation analysis.
\item Conduct statistical evaluation: Apply techniques such as ANOVA F-tests and correlation-based feature selection.
\item Utilize model-driven insights: Interpret feature importance scores obtained from Random Forest models.
\item Reduce dimensional complexity: Implement PCA to support visualization and uncover latent data patterns.

\end{itemize}

\vspace{0.5cm}

\end{document}